The system simulates a population of individuals, named ``blobs", whose sole purpose is reproduction. This is done asexually, when each individual reaches a high enough energy threshold, encoded in his genome. Energy can be gained by eating the food pellets scattered randomly around the world. The ``blobs" eat food by colliding with it, and find it by performing a random walk, in this case, a simplified version of a L\'evy walk\cite{matthaus2009coli}.
The system is made with interactivity in mind, allowing the user to tweak various parameters of the environment, while observing the population evolves in order to adapt to the change. A user could interact more directly with the simulation by adding ``blobs" or placing food. A sudden influx in either will generate a period of instability in the population which eventually reverts back to a stable state.
Information about the population is presented via a graph, showing total number, and average characteristics of the individuals. More specific information about a ``blob" in particular is displayed via a pop-up message, should a creature be clicked on.

\section{Goals and Considerations}
\section{User Interface}
\subsection{User Controls}
\subsection{Data Interface}
\section{Application Design}