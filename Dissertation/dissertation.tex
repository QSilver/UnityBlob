\documentclass{l4proj}

\newcommand{\comment}[1]{\marginpar{\footnotesize
									\color{red}
                                    \textsf{#1}}}

\begin{document}

\title{Blobs on a Plane}

\author{George Popa}

\date{\today}
\maketitle
\begin{abstract}

The software is an evolutionary simulator, dealing with observations of mimicked natural evolutions. The scope of the project is to produce a system that can be used to study evolution in a limited environment where the user can intervene and change the parameters. Particular interest is placed on creature and group behaviour, and population dynamics.

The project involves building a tool which allows visualisation of an environment and enables users to interact with various parameters in the environment. It is intended as an educational tool to allow users to better understand evolutionary dynamics. Creatures called “blobs” evolve in this simulated environment by having techniques used in Genetic Algorithms applied to their DNA. This DNA described their characteristics and behaviour.


\end{abstract}

\tableofcontents

%==============================================================================

\chapter{Introduction}
\label{introduction}
\section{Motivation}
While multiple systems implementing genetic algorithms are available (\ref{prevwork}), they tend to be obsolete, no longer supported, or generally intended as frameworks providing only the algorithms to be used in optimization. The project is intended as a light-weight educational utility, in order to provide insight into the dynamics of an evolving population under environmental conditions. The purpose of the simulation is to observe how average values of parameters change with respect to changes in the environment.

\section{Overview}



\section{Dissertation Outline}
The rest of the dissertation presents in detail the process of development of the evolutionary system, comically named ``Blobs on a Plane". The outline is as follows:

Chapter 2 deals with some of the previous research carried out in the field, as well as other pre-existing applications.

Chapter 3 details the functional and non-functional requirements, as well as the requirements gathering process.

Chapter 4 discusses various concerns about the design of the system and the user interface, any assumptions, and the design decisions made. Presented are also the initial design ideas and the final design.

Chapter 5 describes the implementation details, choice of framework, some of the evolutionary algorithms used, as well as the project deployment.

Chapter 6 deals primarily with evaluation, from both users and experts, as well as internal testing.

Chapter 7 provides a summary of the project, project management matters, and future work.

%=============================================================================

\chapter{Requirements}
\label{requirements}
\section{Requirements Gathering Process}
This section describes the requirements gathering process, as well as enumerates and details the requirements themselves. Since the project is self-defined, there were no requirements provided.
Based on the initial project specification, a list of requirements was drafted. During the background research this list changed both shape and scope. It reached its final form after a video conference with Simon Hickinbotham \cite{simonyork}, a researcher in Evolving Systems. Simon's expertise proved vital in aligning the simulated behaviour to be more in line with real-life bacteria.

\section{Functional / Non-Functional}
The final functional requirements are presented in MoSCoW format \cite{brennan2009guide}. This was selected in order to complement the Agile style of development and to provide clear priorities. As such, critical deliverables are presented as ``Must", important features are labelled as ``Should", desirable requirements as ``Could", and least-critical items as ``Would".
\subsection{Functional Requirements}

\subsubsection{Must Have}
\begin{itemize}
	\item \textbf{Visualisation} \\The system should provide the user with a visualisation of the simulation.
	\item \textbf{Parameter sliders} \\The user should be able to control parameters of the environment.
	\item \textbf{Statistics} \\The user should be provided with details about the population.
	\item \textbf{``Blob" behaviour} \\The system should be capable to simulate food finding behaviour.
	\item \textbf{Evolve blobs} \\The system should evolve the blobs in some way.
\end{itemize}

\subsubsection{Should Have}
\begin{itemize}
	\item \textbf{Direct user interaction} \\The user should be able to actively interact with the simulation in real time. (Add food sources, add extra ``blobs")
	\item \textbf{Reproduction} \\The system should be able to simulate some form of reproduction; either sexual or asexual.
\end{itemize}

\subsubsection{Could Have}
\begin{itemize}
	\item \textbf{Camera controls} \\The user should be able to move the camera around, as well as zoom in or out.
	\item \textbf{Evolutionary parameters} \\The user should be able to interact with the parameters of the evolution itself.
\end{itemize}

\subsubsection{Would Have}
\begin{itemize}
	\item \textbf{Tree crossover} \\The evolution should include tree crossover for blob behaviour.
	\item \textbf{``Blob" characteristics} \\The user should be able to tweak the characteristics of individual ``blobs".
\end{itemize}

\subsection{Non-Functional Requirements}
Since the system's intended purpose is that of a teaching tool, the non-functional requirements are concerned mainly with usability and interactivity. While performance is not an immediate concern, the systems should be able to deal with a rapidly increasing population.

\begin{itemize}
	\item \textbf{Intuitiveness} \\The user interface should be easy to use and intuitive.
	\item \textbf{Interactivity} \\The user should be able to interact with the simulation.
	\item \textbf{Performance} \\The visualisation should be able to handle a sizeable population.
	\item \textbf{Cross-Platform} \\The system should be able to run on a multitude of platforms.
\end{itemize}

\section{Summary}
This chapter presented the requirements for the project. The next chapter will explain how the system was designed in order to match these requirements.

%=============================================================================

\chapter{Design}
\label{design}
\section{Goals and Considerations}
\section{User Interface}
\subsection{User Controls}
\subsection{Data Interface}
\section{Application Design}

%==============================================================================

\chapter{Implementation}
\label{implementation}
\section{Framework Choice}
\section{Overall Architecture}
\section{Main Logic}
\subsection{Simulator State}
\subsection{Update}
\subsection{Save}
\section{Blob Logic}
\subsection{Characteristics}
\subsection{DNA}
\subsection{Update}
\subsection{Movement}
\subsection{Reproduction}
\section{Deployment}

%==============================================================================

\chapter{Evaluation}
\label{evaluation}
\section{Testing and Deployment}
\section{User Evaluation}
\subsection{Feedback Forms}
\subsection{Focus Group}

%==============================================================================

 \chapter{Conclusion}
\label{conclusion}
\section{Summary}
\section{Project Management}
\section{Future Work}
\section{Reflection}

%==============================================================================

\bibliographystyle{plain}
\bibliography{dissertation}

\label{appendix}
\chapter{User Stories}
I want to be able to see the "blobs" in their environment, so that I am able to study their behaviour.

I want to be able to pan and zoom the display area, so that I am able to better observe the environment.

I want to be able to inspect an individual, so that I am able to predict its behaviour.

I want to be able to pause and restart the simulation, so that I may analyse individuals.

I want to be able to reset the simulation, so that I am able to see different evolutions under the same conditions.

I want to be able to tweak parameters in the environment, so that I am able to influence the evolution.

I want to be able to see statistics about the environment and the simulation, so that I am able to measure the effect of changes.

\chapter{Feedback Form} \label{appendix:b}
asd

%==============================================================================

\end{document}