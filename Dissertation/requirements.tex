\section{Requirements Gathering Process}
This section describes the requirements gathering process, as well as enumerates and details the requirements themselves. Since the project is self-defined, there were no requirements provided.
Based on the initial project specification, a list of requirements was drafted. During the background research this list changed both shape and scope. It reached its final form after a video conference with Simon Hickinbotham \cite{simonyork}, a researcher in Evolving Systems. Simon's expertise proved vital in aligning the simulated behaviour to be more in line with real-life bacteria.

\section{Functional / Non-Functional}
The final functional requirements are presented in MoSCoW format \cite{brennan2009guide}. This was selected in order to complement the Agile style of development and to provide clear priorities. As such, critical deliverables are presented as ``Must", important features are labelled as ``Should", desirable requirements as ``Could", and least-critical items as ``Would".
\subsection{Functional Requirements}

\subsubsection{Must Have}
\begin{itemize}
	\item \textbf{Visualisation} \\The system should provide the user with a visualisation of the simulation.
	\item \textbf{Parameter sliders} \\The user should be able to control parameters of the environment.
	\item \textbf{Statistics} \\The user should be provided with details about the population.
	\item \textbf{``Blob" behaviour} \\The system should be capable to simulate food finding behaviour.
	\item \textbf{Evolve blobs} \\The system should evolve the blobs in some way.
\end{itemize}

\subsubsection{Should Have}
\begin{itemize}
	\item \textbf{Direct user interaction} \\The user should be able to actively interact with the simulation in real time. (Add food sources, add extra ``blobs")
	\item \textbf{Reproduction} \\The system should be able to simulate some form of reproduction; either sexual or asexual.
\end{itemize}

\subsubsection{Could Have}
\begin{itemize}
	\item \textbf{Camera controls} \\The user should be able to move the camera around, as well as zoom in or out.
	\item \textbf{Evolutionary parameters} \\The user should be able to interact with the parameters of the evolution itself.
\end{itemize}

\subsubsection{Would Have}
\begin{itemize}
	\item \textbf{Tree crossover} \\The evolution should include tree crossover for blob behaviour.
	\item \textbf{``Blob" characteristics} \\The user should be able to tweak the characteristics of individual ``blobs".
\end{itemize}

\subsection{Non-Functional Requirements}
Since the system's intended purpose is that of a teaching tool, the non-functional requirements are concerned mainly with usability and interactivity. While performance is not an immediate concern, the systems should be able to deal with a rapidly increasing population.

\begin{itemize}
	\item \textbf{Intuitiveness} \\The user interface should be easy to use and intuitive.
	\item \textbf{Interactivity} \\The user should be able to interact with the simulation.
	\item \textbf{Performance} \\The visualisation should be able to handle a sizeable population.
	\item \textbf{Cross-Platform} \\The system should be able to run on a multitude of platforms.
\end{itemize}

\section{Summary}
This chapter presented the requirements for the project. The next chapter will explain how the system was designed in order to match these requirements.